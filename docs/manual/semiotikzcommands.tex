\usepackage{fontspec}
\newfontfamily\NotoEmoji{Noto Emoji}

\usepackage{amsmath}
%For \CIRCLE
\usepackage{wasysym}


\usepackage{tikz,pgfplots}
\usetikzlibrary{
	positioning,
	shapes.geometric, shapes.callouts,
	external,
	graphs, graphdrawing,
	trees,
	backgrounds,
	fit,
	3d,
	svg.path}
\usegdlibrary{trees, layered}

%\tikzset{external/system call={lualatex -enable-write18 -halt-on-error -interaction=batchmode -jobname "\image" "\texsource"}}
% \tikzset{external/system call={lualatex -shell-escape -halt-on-error -interaction=batchmode -jobname "\image" "\texsource"}}
% \tikzexternalize[prefix=figures/]

\usepackage{forest}
\useforestlibrary{edges}
% For using forked edge style
\ProvidesForestLibrary{edges}[2016/12/05 v0.1.1]

% \newcommand{\inputtikz}[1]{\tikzsetnextfilename{#1}\input{#1.tikz.tex}}

% \usepackage[dvipsnames]{xcolor}
\definecolor{semioturquoise}{HTML}{00A69D}
\definecolor{semiopink}{HTML}{FF344F}
\definecolor{semioblue}{HTML}{334fea}
\definecolor{semioyellow}{HTML}{eacf33}
\definecolor{semioorange}{RGB}{240,117,24}
\definecolor{semiopurple}{HTML}{9d00a5}
\definecolor{semioanthrazite}{RGB}{75,85,100}
\definecolor{semioanthraziteLite}{RGB}{191,216,255}
\definecolor{semioanthraziteDark}{RGB}{37,42,50}

\tikzset{
    sobject/.style={draw,circle,node font=\normalsize},
    sobjectDescription/.style={dotted,draw,rectangle,align=left,node font=\tiny},
    sobjectDescriptionEdge/.style={dotted,node font=\tiny}
	% Port style causes infinite loop somehow...    
    %port/.style={sloped,anchor=south,node font=\tiny}
}

\usepackage{ifthen}

%---------------------------------------SYMBOLS---------------------------------------

\newcommand{\emoji}[1]{{\NotoEmoji{#1}}}
% math emoji
\newcommand{\memoji}[1]{{\text{\emoji{#1}}}}

\newcommand{\sTabSpace}{0.15cm}
\newcommand{\sTabs}[1]{\foreach \s in {1,...,#1} {\hspace{\sTabSpace}}}


\def\semioname{Semio}
\def\semioversion{0.2.0}

\newcommand{\sSobjectWithDescriptionDescriptionSpace}{0.1cm}

\newcommand{\sUri}{\emoji{🔗}}

\newcommand{\sDefinition}{\emoji{🏗️}}
\newcommand{\sPlan}{\emoji{🛠️}}
\newcommand{\sPort}{\emoji{🪝}}

%---------------------------------------GENERAL PURPOSE FUNCIONS---------------------------------------

%inputs: cardinal direction
\newcommand{\cardianalToDirection}[1] {\ifthenelse{\equal{#1}{north}}{above}{\ifthenelse{\equal{#1}{east}}{right}{\ifthenelse{\equal{#1}{south}}{below}{left}}}}

%inputs: direction
\newcommand{\directionToCardinal}[1] {\ifthenelse{\equal{#1}{above}}{north}{\ifthenelse{\equal{#1}{right}}{east}{\ifthenelse{\equal{#1}{below}}{south}{west}}}}

%inputs: cardinal direction
\newcommand{\invertCardianal}[1] {\ifthenelse{\equal{#1}{north}}{south}{\ifthenelse{\equal{#1}{east}}{west}{\ifthenelse{\equal{#1}{south}}{north}{east}}}}

%inputs: direction
\newcommand{\invertDirection}[1] {\ifthenelse{\equal{#1}{above}}{below}{\ifthenelse{\equal{#1}{right}}{left}{\ifthenelse{\equal{#1}{below}}{above}{right}}}}

%---------------------------------------SPECIAL PURPOSE FUNCIONS---------------------------------------

%inputs: uri,parameters[], attributes[]
\newcommand{\sSobjectWithDescriptionDescription}[1]{\sDefinition:\\\sTabs{1}\sPlan:\\\sTabs{2}\sUri=#1}

\newcommand{\sSimpleDefinitionDescription}[1]{\sSobjectWithDescriptionDescription{#1}}

%inputs: id, location attributes, description attributes 
\newcommand{\sSobjectWithDescription}[5]{
	\node [sobject] (#1) [#2] {#5};
	\node [sobjectDescription] (#1D)
		[rectangle callout,callout absolute pointer={(#1)},
		#3=\sSobjectWithDescriptionDescriptionSpace of #1,#4]
		{\sSimpleDefinitionDescription{base}};}

%inputs: id, location attributes, display
\newcommand{\sSobject}[3]{
	\node [sobject] (#1) [#2] {#3};}

%inputs: connectedId, connectingId
\newcommand{\sConnection}[2]{
	\draw (#1) -- (#2);}

%inputs: connectedId, connectingId
\newcommand{\sConnectionWithPorts}[6]{
	\draw (#1) -- (#2)
		% port style doesn't work: sloped,anchor=south,node font=\tiny
		node[sloped,anchor=south,node font=\tiny, at start,anchor=#5]{\sPort: #3}
		node[sloped,anchor=south,node font=\tiny, at end,anchor=#6]{\sPort: #4};}

%inputs: connectedId, connectingId, connectedPorts, connectingPorts, connectedPortSide: north|south, connectingPortSide: north|south
\newcommand{\sConnectionWithPortsBroken}[6]{
	\draw (#1) -- (#2)
		node[port,at start,anchor=#5]{\sPort: #3}
		node[port,at end,anchor=#6]{\sPort: #4};}