\usepackage{fontspec}
\newfontfamily\NotoEmoji{Noto Emoji}
% \newfontfamily\NotoEmoji{../fonts/notoemoji/static/NotoEmoji-Regular.ttf}


\usepackage{amsmath}
%For \CIRCLE
\usepackage{wasysym}

\usepackage{varwidth}

\usepackage{tikz,pgfplots}
\usetikzlibrary{
	positioning,
	shapes.geometric, shapes.callouts, shapes.misc,
	external,
	graphs, graphdrawing,
	trees,
	backgrounds,
	fit,
	3d,
	svg.path,
	decorations.pathreplacing,
	calligraphy}
\usegdlibrary{trees, layered}

%\tikzset{external/system call={lualatex -enable-write18 -halt-on-error -interaction=batchmode -jobname "\image" "\texsource"}}
% \tikzset{external/system call={lualatex -shell-escape -halt-on-error -interaction=batchmode -jobname "\image" "\texsource"}}
% \tikzexternalize[prefix=figures/,shell escape=-enable-write18]
% \tikzexternalize[prefix=figures/]


% \usepackage{forest}
% \useforestlibrary{edges}
% % For using forked edge style
% \ProvidesForestLibrary{edges}[2016/12/05 v0.1.1]

% % For scaling tikzpictures to a width
% % https://tex.stackexchange.com/questions/481915/how-can-i-scale-tikz-picture-to-full-width
% % https://tex.stackexchange.com/questions/6388/how-to-scale-a-tikzpicture-to-textwidth
% \usepackage{environ}
% \newsavebox{\measure@tikzpicture}
% \NewEnviron{scaletikzpicturetowidth}[1]{
%   \def\tikz@width{#1}
%   \def\tikzscale{1}\begin{lrbox}{\measure@tikzpicture}
%   \BODY
%   \end{lrbox}
%   \pgfmathparse{#1/\wd\measure@tikzpicture}
%   \edef\tikzscale{\pgfmathresult}
%   \BODY
% }

% \newcommand{\inputtikz}[1]{\tikzsetnextfilename{#1}\input{#1.tikz.tex}}

% \usepackage[dvipsnames]{xcolor}
\definecolor{semioPrimaryColor}{HTML}{FF344F}
\definecolor{semioSecondaryColor}{HTML}{00A69D}
\definecolor{semioTertiaryColor}{HTML}{FA9500}
\definecolor{semioQuaternaryColor}{HTML}{415A77}
\definecolor{semioQuintaryColor}{HTML}{7C6A0A}
\definecolor{semioSenaryColor}{HTML}{A799B7}
\definecolor{semioPrimaryAccentColor}{HTML}{DCED31}
\definecolor{semioSecondaryAccentColor}{HTML}{FCCF05}
\definecolor{semioLightColor}{HTML}{F7F3E3}
\definecolor{semioDarkColor}{HTML}{0A2E36}

\tikzset{
	semioGeneralTerm/.style={draw,loosely dashed},
	semioTerm/.style={draw,dashed},
	semioSpecificTerm/.style={draw,densely dashed},
	semioGeneralItem/.style={draw,loosely dotted},
	semioItem/.style={draw,dotted},
	semioSpecificItem/.style={draw,densely dotted},
	semioTermText/.style={node font={\large}},% Somehow \textsc doesn't work
	semioBoundingBox/.style={rounded corners},
    sobject/.style={rectangle,node font=\normalsize,minimum size=0,inner sep=0pt},
	unknownSobject/.style={draw,circle,minimum size=1em},
    sobjectDescription/.style={rectangle,align=left,node font=\tiny},
	attractionEdge/.style={->},  
    port/.style={sloped,anchor=south,node font=\tiny},
	knotPort/.style={},
	new/.style={semioPrimaryColor},
	newSobject/.style={semioPrimaryColor,fill=white},
	newConnection/.style={semioPrimaryColor},
	newDesign/.style={semioPrimaryColor},
	old/.style={black},
	modified/.style={semioPrimaryAccentColor},
	semioScheme/.style={draw=black,dotted},
	semioSchemeAlgorithm/.style={draw,rectangle,node font=\normalsize,minimum size=0,inner sep=3pt},
	semioRule/.style={draw=black,dotted},
	ruleCopy/.style={->},
	ruleInput/.style={semioSecondaryColor,<-},
	ruleOutput/.style={semioSecondaryColor,->},
	resolvesTo/.style={semioQuaternaryColor,->},
	semioLayout/.style={draw=black,dotted},
	semioDesign/.style={draw=black,dotted},
	scriptCall/.style={semioSenaryColor,->},
	connection/.style={--},
	scriptLegend/.style={node font=\scriptsize},
	definitionDrawing/.style={draw,dotted},
    definitionDrawingHidden/.style={draw,densely dotted},
    % Somehow doesn't do anything: label distance=0
    definitionDrawingPointAnnotation/.style={draw,cross out,semioSecondaryAccentColor,font=\tiny,minimum size=0,inner sep=1pt},
    definitionDrawingLinearAnnotation/.style={semioSecondaryAccentColor,|<->|,font=\tiny,minimum size=0,inner sep=1pt},
    % curly brace doesn't work as style somehow
	definitionDrawingCountAnnotation/.style={decorate,decoration={brace,mirror},semioSecondaryAccentColor,font=\tiny},
	definitionDrawingCurveAnnotation/.style={semioSecondaryAccentColor,font=\tiny,minimum size=0,inner sep=1pt},
    definitionDrawingTextAnnotation/.style={semioSecondaryAccentColor,font=\tiny,minimum size=0,inner sep=1pt},
    definitionDrawingPort/.style={semioTertiaryColor,font=\tiny,minimum size=0,inner sep=0pt},
    definitionDrawingPortAxis/.style={semioTertiaryColor,->,minimum size=0,inner sep=1pt},
    definitionDrawingPortAxisText/.style={semioTertiaryColor,font=\tiny}
}

%inputs: nodeNamePrefix,boxStyle, xShift, yShift, additionalScopeStyle
\newenvironment{boundingBox}[5]
	{\def\nodeNamePrefixAuxiallaryVariable{#1}
	\def\boxStyleAuxiallaryVariable{#2}
	\begin{scope}[local bounding box/.expanded=\nodeNamePrefixAuxiallaryVariable-lBB,xshift=#3,yshift=#4,#5]}
	{\end{scope}
	\node [semioBoundingBox, fit=(\nodeNamePrefixAuxiallaryVariable-lBB),\boxStyleAuxiallaryVariable](\nodeNamePrefixAuxiallaryVariable-bB) {};}

%inputs: nodeNamePrefix, labelContent, boxStyle, labelStyle, xShift, yShift, additionalScopeStyle
\newenvironment{boundingBoxWithLabel}[7]
	{\def\nodeNamePrefixAuxiallaryVariable{#1}
	\def\labelAuxiallaryVariable{#2}
	\def\boxStyleAuxiallaryVariable{#3}
 	\def\labelStyleAuxiallaryVariable{#4}
	\begin{scope}[local bounding box/.expanded=\nodeNamePrefixAuxiallaryVariable-lBB,xshift=#5,yshift=#6,#7]}
	{\end{scope}
	\node [semioBoundingBox, fit=(\nodeNamePrefixAuxiallaryVariable-lBB),
		label={[anchor=south west,name=\nodeNamePrefixAuxiallaryVariable-bBL,align=left,\labelStyleAuxiallaryVariable]north west:{\labelAuxiallaryVariable}},\boxStyleAuxiallaryVariable] 
		(\nodeNamePrefixAuxiallaryVariable-bB) {};
	\coordinate[yshift=-\fitSpacing] (\nodeNamePrefixAuxiallaryVariable-bBTM) at (\nodeNamePrefixAuxiallaryVariable-bB.north);
	\node [fit=(\nodeNamePrefixAuxiallaryVariable-bB)(\nodeNamePrefixAuxiallaryVariable-bBTM)] (\nodeNamePrefixAuxiallaryVariable-bBT) {};}

\usepackage{ifthen}

\usepackage{listings}
\lstdefinestyle{schemeStyle}{
	backgroundcolor=\color{semioLightColorLite},   
	numberstyle=\tiny,
	breakatwhitespace=false,         
	breaklines=true,                 
	captionpos=b,                    
	keepspaces=true,                 
	numbers=left,                    
	numbersep=5pt,                  
	showspaces=false,                
	showstringspaces=false,
	showtabs=false,                  
	tabsize=2
}

\lstset{style=schemeStyle}


%---------------------------------------SYMBOLS---------------------------------------

\newcommand{\emoji}[1]{{\NotoEmoji{#1}}}
% math emoji
\newcommand{\memoji}[1]{{\text{\emoji{#1}}}}

\newcommand{\sTabSpace}{0.15cm}
\newcommand{\sTabs}[1]{\foreach \s in {1,...,#1} {\hspace{\sTabSpace}}}
\newcommand{\horizontalSpacing}{0.2}
\newcommand{\verticalSpacing}{0.2}
\newcommand{\fitSpacing}{0.125}

\def\semioname{Semio}
\def\semioversion{0.2.0}

\newcommand{\sobjectWithDescriptionDescriptionSpace}{0.1cm}

\newcommand{\uriIcon}{\emoji{🔗}}
\newcommand{\attributeIcon}{\emoji{🏷️}}
\newcommand{\attributeEqual}[2]{$\text{\attributeIcon}[\memoji{#1}]=#2$}
\newcommand{\attributeEqualSame}[1]{$\text{\attributeIcon}[\memoji{#1}]$}
\newcommand{\attributeAssign}[2]{$\text{\attributeIcon}[\memoji{#1}]:#2$}
\newcommand{\attributeAssignSame}[1]{$\text{\attributeIcon}[\memoji{#1}]$}

\newcommand{\prototypeIcon}{□}
\newcommand{\definitionIcon}{\emoji{🏗️}}
\newcommand{\planIcon}{\emoji{🛠️}}
\newcommand{\parameterEqual}[2]{$\text{\planIcon}[\memoji{#1}]=#2$}
\newcommand{\parameterAssign}[2]{$\text{\planIcon}[\memoji{#1}]:#2$}
\newcommand{\parameterAssignSame}[1]{$\text{\planIcon}[\memoji{#1}]$}


\newcommand{\portIcon}{\emoji{🪝}}
\newcommand{\transformationArrow}{\Rightarrow}

%---------------------------------------GENERAL PURPOSE FUNCIONS---------------------------------------

%inputs: cardinal direction
\newcommand{\cardianalToDirection}[1] {\ifthenelse{\equal{#1}{north}}{above}{\ifthenelse{\equal{#1}{east}}{right}{\ifthenelse{\equal{#1}{south}}{below}{left}}}}

%inputs: direction
\newcommand{\directionToCardinal}[1] {\ifthenelse{\equal{#1}{above}}{north}{\ifthenelse{\equal{#1}{right}}{east}{\ifthenelse{\equal{#1}{below}}{south}{west}}}}

%inputs: cardinal direction
\newcommand{\invertCardianal}[1] {\ifthenelse{\equal{#1}{north}}{south}{\ifthenelse{\equal{#1}{east}}{west}{\ifthenelse{\equal{#1}{south}}{north}{east}}}}

%inputs: direction
\newcommand{\invertDirection}[1] {\ifthenelse{\equal{#1}{above}}{below}{\ifthenelse{\equal{#1}{right}}{left}{\ifthenelse{\equal{#1}{below}}{above}{right}}}}

%---------------------------------------SPECIAL PURPOSE FUNCIONS---------------------------------------

%inputs: nodeId, uri, prototype, displayId, location attributes, additionalStyle
\newcommand{\sobject}[6]{
	\node [sobject,#5,#6] (#1) {$\text{#2}_{\text{#4}}^{\text{#3}}$};}
	
%inputs: nodeId, uri, prototype, displayId, location attributes, content, description attributes
\newcommand{\sobjectWithDescription}[9]{
	\sobject{#1}{#2}{#3}{#4}{#5}{#6}
	\node [semioSpecificItem,sobjectDescription] (#1D)
		[rectangle callout,callout absolute pointer={(#1)},
		#7=\sobjectWithDescriptionDescriptionSpace of #1,#9]
		{#8};}

%inputs: attractingId, attractedId
\newcommand{\sAttraction}[2]{
	\draw (#1) [attractionEdge]-- (#2);}

%inputs: attractingId, attractedId, knots, attractedPorts, attractingPorts, id, knotAnchor, attractingAnchor, attractedAnchor, idAnchor
%WARNING: 10th parameter needs to be passed over helper variable: \def\tempVarAAttractionWithPorts{idAnchor}
%Latex only accepts up to 9 parameters...
\newcommand{\sAttractionWithPorts}[9]{
	\begin{pgfonlayer}{ports}
		\draw (#1) [attractionEdge]-- (#2)
			node[port, at start,anchor=#7]{#4} %port attracting
			node[port, midway,anchor=center]{#3} %knots
			node[port, at start,anchor=#9]{#6} %id
			node[port, at end,anchor=#8]{#5}; %port attracted
	\end{pgfonlayer}}

%inputs: transformationName, transformationLabel, xCoordinateLHS, yCoordinateLHS,xCoordinateRHS, yCoordinateRHS,
\newcommand{\sTransformation}[6]{
	\begin{boundingBoxWithLabel}{trf-#1}{#2}{semioSpecificTerm}{}{0}{0}{}
		\begin{boundingBox}{trf-l-#1}{semioItem}{0}{0}{}
			\pic (trf-l-#1) at (#3,#4) {trf-l-#1};
		\end{boundingBox}
		\begin{boundingBox}{trf-r-#1}{semioItem}{0}{0}{}
			\pic (trf-r-#1) at (#5,#6) {trf-r-#1};
		\end{boundingBox}
	\end{boundingBoxWithLabel}
	\draw (trf-l-#1-bB.east)[draw=none]--(trf-r-#1-bB.west) node [midway] {\transformationArrow};}

%inputs: modificationName, modificationLabel, parameter, xCoordinateLHS, yCoordinateLHS, xCoordinateRHS, yCoordinateRHS, scaling factor
\newcommand{\sModification}[8]{
	\begin{boundingBoxWithLabel}{mod-#1}{#2}{semioSpecificTerm}{}{0}{0}{}
		\begin{boundingBox}{mod-o-#1}{semioItem}{0}{0}{}
			\pic[scale=#8] (mod-o-#1) at (#4,#5) {mod-o-#1};
		\end{boundingBox}
		\begin{boundingBox}{mod-m-#1}{semioItem}{0}{0}{}
			\pic[scale=#8] (mod-m-#1) at (#6,#7) {mod-m-#1};
		\end{boundingBox}
	\end{boundingBoxWithLabel}
	\draw (mod-o-#1-bB.east)[draw=none]--(mod-m-#1-bB.west) node [midway,label={[anchor=south]north:{#3}}] {$\transformationArrow$};}

%inputs: attractingNodeId, attractedNodeId, attractingPort, attractedPort, distanceBetweenNodes, portStyle
\newcommand{\sKnot}[6]{
	\sobject{#1}{}{}{}{}{}
	\sobject{#2}{}{}{}{right=#5 of knts-ud-l}{}
	\draw (#1) [attractionEdge]-- (#2)
		node[knotPort,at start,anchor=south west,#6]{#3} %port attracting
		node[knotPort,at end,anchor=south east,#6]{#4};} %port attracted

%https://tex.stackexchange.com/questions/609638/tikz-local-bounding-box-in-nested-scopes
% Auxillary variable as workaround to pass argument into custom environment command
% https://tex.stackexchange.com/questions/380277/custom-environment-with-minipages-causes-illegal-parameter-number-in-definition

\pgfmathsetmacro\designlevel{0}
\newenvironment{design}[0]
{\pgfmathsetmacro\designlevel{int(\designlevel+1)}
\begin{scope}[local bounding box/.expanded=bounding box \designlevel,sharp corners]}
{\end{scope}\node [semioDesign,fit=(bounding box \designlevel)] (bB) {};}

\newenvironment{designLast}[0]
{\begin{scope}[local bounding box/.expanded=bounding box designLastlevel,sharp corners]}
{\end{scope}\node [semioTerm,fit=(bounding box designLastlevel)] (bB) {};}
		
\pgfmathsetmacro\definitionLevel{0}
\newenvironment{definition}[1]
  { \def\labelNameAuxiallaryVariable{#1}
    \pgfmathsetmacro\definitionLevel{int(\definitionLevel+1)}
  \begin{scope}[local bounding box/.expanded=bounding box \definitionLevel,sharp corners]}
  {\end{scope}
    \node [semioSpecificTerm,fit=(bounding box \definitionLevel),
      label={[semioTermText, anchor=north east]north east:{\labelNameAuxiallaryVariable}}] 
      (bB) {};}
